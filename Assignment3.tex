\PassOptionsToPackage{unicode=true}{hyperref} % options for packages loaded elsewhere
\PassOptionsToPackage{hyphens}{url}
%
\documentclass[]{article}
\usepackage{lmodern}
\usepackage{amssymb,amsmath}
\usepackage{ifxetex,ifluatex}
\usepackage{fixltx2e} % provides \textsubscript
\ifnum 0\ifxetex 1\fi\ifluatex 1\fi=0 % if pdftex
  \usepackage[T1]{fontenc}
  \usepackage[utf8]{inputenc}
  \usepackage{textcomp} % provides euro and other symbols
\else % if luatex or xelatex
  \usepackage{unicode-math}
  \defaultfontfeatures{Ligatures=TeX,Scale=MatchLowercase}
\fi
% use upquote if available, for straight quotes in verbatim environments
\IfFileExists{upquote.sty}{\usepackage{upquote}}{}
% use microtype if available
\IfFileExists{microtype.sty}{%
\usepackage[]{microtype}
\UseMicrotypeSet[protrusion]{basicmath} % disable protrusion for tt fonts
}{}
\IfFileExists{parskip.sty}{%
\usepackage{parskip}
}{% else
\setlength{\parindent}{0pt}
\setlength{\parskip}{6pt plus 2pt minus 1pt}
}
\usepackage{hyperref}
\hypersetup{
            pdftitle={Advanced Macro},
            pdfauthor={Hans Martinez},
            pdfborder={0 0 0},
            breaklinks=true}
\urlstyle{same}  % don't use monospace font for urls
\usepackage[margin=1in]{geometry}
\usepackage{color}
\usepackage{fancyvrb}
\newcommand{\VerbBar}{|}
\newcommand{\VERB}{\Verb[commandchars=\\\{\}]}
\DefineVerbatimEnvironment{Highlighting}{Verbatim}{commandchars=\\\{\}}
% Add ',fontsize=\small' for more characters per line
\usepackage{framed}
\definecolor{shadecolor}{RGB}{248,248,248}
\newenvironment{Shaded}{\begin{snugshade}}{\end{snugshade}}
\newcommand{\AlertTok}[1]{\textcolor[rgb]{0.94,0.16,0.16}{#1}}
\newcommand{\AnnotationTok}[1]{\textcolor[rgb]{0.56,0.35,0.01}{\textbf{\textit{#1}}}}
\newcommand{\AttributeTok}[1]{\textcolor[rgb]{0.77,0.63,0.00}{#1}}
\newcommand{\BaseNTok}[1]{\textcolor[rgb]{0.00,0.00,0.81}{#1}}
\newcommand{\BuiltInTok}[1]{#1}
\newcommand{\CharTok}[1]{\textcolor[rgb]{0.31,0.60,0.02}{#1}}
\newcommand{\CommentTok}[1]{\textcolor[rgb]{0.56,0.35,0.01}{\textit{#1}}}
\newcommand{\CommentVarTok}[1]{\textcolor[rgb]{0.56,0.35,0.01}{\textbf{\textit{#1}}}}
\newcommand{\ConstantTok}[1]{\textcolor[rgb]{0.00,0.00,0.00}{#1}}
\newcommand{\ControlFlowTok}[1]{\textcolor[rgb]{0.13,0.29,0.53}{\textbf{#1}}}
\newcommand{\DataTypeTok}[1]{\textcolor[rgb]{0.13,0.29,0.53}{#1}}
\newcommand{\DecValTok}[1]{\textcolor[rgb]{0.00,0.00,0.81}{#1}}
\newcommand{\DocumentationTok}[1]{\textcolor[rgb]{0.56,0.35,0.01}{\textbf{\textit{#1}}}}
\newcommand{\ErrorTok}[1]{\textcolor[rgb]{0.64,0.00,0.00}{\textbf{#1}}}
\newcommand{\ExtensionTok}[1]{#1}
\newcommand{\FloatTok}[1]{\textcolor[rgb]{0.00,0.00,0.81}{#1}}
\newcommand{\FunctionTok}[1]{\textcolor[rgb]{0.00,0.00,0.00}{#1}}
\newcommand{\ImportTok}[1]{#1}
\newcommand{\InformationTok}[1]{\textcolor[rgb]{0.56,0.35,0.01}{\textbf{\textit{#1}}}}
\newcommand{\KeywordTok}[1]{\textcolor[rgb]{0.13,0.29,0.53}{\textbf{#1}}}
\newcommand{\NormalTok}[1]{#1}
\newcommand{\OperatorTok}[1]{\textcolor[rgb]{0.81,0.36,0.00}{\textbf{#1}}}
\newcommand{\OtherTok}[1]{\textcolor[rgb]{0.56,0.35,0.01}{#1}}
\newcommand{\PreprocessorTok}[1]{\textcolor[rgb]{0.56,0.35,0.01}{\textit{#1}}}
\newcommand{\RegionMarkerTok}[1]{#1}
\newcommand{\SpecialCharTok}[1]{\textcolor[rgb]{0.00,0.00,0.00}{#1}}
\newcommand{\SpecialStringTok}[1]{\textcolor[rgb]{0.31,0.60,0.02}{#1}}
\newcommand{\StringTok}[1]{\textcolor[rgb]{0.31,0.60,0.02}{#1}}
\newcommand{\VariableTok}[1]{\textcolor[rgb]{0.00,0.00,0.00}{#1}}
\newcommand{\VerbatimStringTok}[1]{\textcolor[rgb]{0.31,0.60,0.02}{#1}}
\newcommand{\WarningTok}[1]{\textcolor[rgb]{0.56,0.35,0.01}{\textbf{\textit{#1}}}}
\usepackage{longtable,booktabs}
% Fix footnotes in tables (requires footnote package)
\IfFileExists{footnote.sty}{\usepackage{footnote}\makesavenoteenv{longtable}}{}
\usepackage{graphicx,grffile}
\makeatletter
\def\maxwidth{\ifdim\Gin@nat@width>\linewidth\linewidth\else\Gin@nat@width\fi}
\def\maxheight{\ifdim\Gin@nat@height>\textheight\textheight\else\Gin@nat@height\fi}
\makeatother
% Scale images if necessary, so that they will not overflow the page
% margins by default, and it is still possible to overwrite the defaults
% using explicit options in \includegraphics[width, height, ...]{}
\setkeys{Gin}{width=\maxwidth,height=\maxheight,keepaspectratio}
\setlength{\emergencystretch}{3em}  % prevent overfull lines
\providecommand{\tightlist}{%
  \setlength{\itemsep}{0pt}\setlength{\parskip}{0pt}}
\setcounter{secnumdepth}{0}
% Redefines (sub)paragraphs to behave more like sections
\ifx\paragraph\undefined\else
\let\oldparagraph\paragraph
\renewcommand{\paragraph}[1]{\oldparagraph{#1}\mbox{}}
\fi
\ifx\subparagraph\undefined\else
\let\oldsubparagraph\subparagraph
\renewcommand{\subparagraph}[1]{\oldsubparagraph{#1}\mbox{}}
\fi

% set default figure placement to htbp
\makeatletter
\def\fps@figure{htbp}
\makeatother

\usepackage{etoolbox}
\makeatletter
\providecommand{\subtitle}[1]{% add subtitle to \maketitle
  \apptocmd{\@title}{\par {\large #1 \par}}{}{}
}
\makeatother

\title{Advanced Macro}
\providecommand{\subtitle}[1]{}
\subtitle{Assignment 3}
\author{Hans Martinez}
\date{Oct 06, 2020}

\begin{document}
\maketitle

\hypertarget{interpolation}{%
\subsection{Interpolation}\label{interpolation}}

\href{https://github.com/hans-mtz/AdvMacro/blob/master/A3.jl}{Julia
code: click here.}

\begin{Shaded}
\begin{Highlighting}[]
\CommentTok{# Julia code}
\CommentTok{# See A3.jl}
\end{Highlighting}
\end{Shaded}

\hypertarget{newton-basis-polynomial-interpolation}{%
\subsubsection{Newton basis polynomial
interpolation}\label{newton-basis-polynomial-interpolation}}

\begin{longtable}[]{@{}rrrr@{}}
\toprule
\begin{minipage}[b]{0.10\columnwidth}\raggedleft
Function\strut
\end{minipage} & \begin{minipage}[b]{0.26\columnwidth}\raggedleft
n=4\strut
\end{minipage} & \begin{minipage}[b]{0.26\columnwidth}\raggedleft
n=6\strut
\end{minipage} & \begin{minipage}[b]{0.26\columnwidth}\raggedleft
n=11\strut
\end{minipage}\tabularnewline
\midrule
\endhead
\begin{minipage}[t]{0.10\columnwidth}\raggedleft
Log\strut
\end{minipage} & \begin{minipage}[t]{0.26\columnwidth}\raggedleft
\(0.0009543\)\strut
\end{minipage} & \begin{minipage}[t]{0.26\columnwidth}\raggedleft
\(9.481 \cdot 10^{-06}\)\strut
\end{minipage} & \begin{minipage}[t]{0.26\columnwidth}\raggedleft
\(1.391 \cdot 10^{-11}\)\strut
\end{minipage}\tabularnewline
\begin{minipage}[t]{0.10\columnwidth}\raggedleft
Square\strut
\end{minipage} & \begin{minipage}[t]{0.26\columnwidth}\raggedleft
\(0.001947\)\strut
\end{minipage} & \begin{minipage}[t]{0.26\columnwidth}\raggedleft
\(1.934 \cdot 10^{-05}\)\strut
\end{minipage} & \begin{minipage}[t]{0.26\columnwidth}\raggedleft
\(2.838 \cdot 10^{-11}\)\strut
\end{minipage}\tabularnewline
\begin{minipage}[t]{0.10\columnwidth}\raggedleft
CES \(\sigma=2\)\strut
\end{minipage} & \begin{minipage}[t]{0.26\columnwidth}\raggedleft
\(0.0006884\)\strut
\end{minipage} & \begin{minipage}[t]{0.26\columnwidth}\raggedleft
\(6.839 \cdot 10^{-06}\)\strut
\end{minipage} & \begin{minipage}[t]{0.26\columnwidth}\raggedleft
\(1.003 \cdot 10^{-11}\)\strut
\end{minipage}\tabularnewline
\begin{minipage}[t]{0.10\columnwidth}\raggedleft
CES \(\sigma=5\)\strut
\end{minipage} & \begin{minipage}[t]{0.26\columnwidth}\raggedleft
\(2.151 \cdot 10^{-05}\)\strut
\end{minipage} & \begin{minipage}[t]{0.26\columnwidth}\raggedleft
\(2.137 \cdot 10^{-07}\)\strut
\end{minipage} & \begin{minipage}[t]{0.26\columnwidth}\raggedleft
\(3.135 \cdot 10^{-13}\)\strut
\end{minipage}\tabularnewline
\begin{minipage}[t]{0.10\columnwidth}\raggedleft
CES \(\sigma=10\)\strut
\end{minipage} & \begin{minipage}[t]{0.26\columnwidth}\raggedleft
\(2.988 \cdot 10^{-07}\)\strut
\end{minipage} & \begin{minipage}[t]{0.26\columnwidth}\raggedleft
\(2.968 \cdot 10^{-09}\)\strut
\end{minipage} & \begin{minipage}[t]{0.26\columnwidth}\raggedleft
\(4.355 \cdot 10^{-15}\)\strut
\end{minipage}\tabularnewline
\bottomrule
\end{longtable}

Table 1. Upper bound error for polynomial interpolation.

\begin{figure}

{\centering \includegraphics[width=0.5\linewidth,height=0.2\textheight]{Assignment3/graphs/Newton Log _n_4} \includegraphics[width=0.5\linewidth,height=0.2\textheight]{Assignment3/graphs/Newton Log _n_6} \includegraphics[width=0.5\linewidth,height=0.2\textheight]{Assignment3/graphs/Newton Log _n_11} \includegraphics[width=0.5\linewidth,height=0.2\textheight]{Assignment3/graphs/Newton Log _n_21} 

}

\caption{Interpolation Newton Log Fn}\label{fig:unnamed-chunk-2}
\end{figure}

\begin{figure}

{\centering \includegraphics[width=0.5\linewidth,height=0.2\textheight]{Assignment3/graphs/Newton Square _n_4} \includegraphics[width=0.5\linewidth,height=0.2\textheight]{Assignment3/graphs/Newton Square _n_6} \includegraphics[width=0.5\linewidth,height=0.2\textheight]{Assignment3/graphs/Newton Square _n_11} \includegraphics[width=0.5\linewidth,height=0.2\textheight]{Assignment3/graphs/Newton Square _n_21} 

}

\caption{Interpolation Newton Root Square Fn}\label{fig:unnamed-chunk-3}
\end{figure}

\begin{figure}

{\centering \includegraphics[width=0.5\linewidth,height=0.2\textheight]{Assignment3/graphs/Newton CES σ=2 _n_4} \includegraphics[width=0.5\linewidth,height=0.2\textheight]{Assignment3/graphs/Newton CES σ=2 _n_6} \includegraphics[width=0.5\linewidth,height=0.2\textheight]{Assignment3/graphs/Newton CES σ=2 _n_11} \includegraphics[width=0.5\linewidth,height=0.2\textheight]{Assignment3/graphs/Newton CES σ=2 _n_21} 

}

\caption{Interpolation Newton CES sigma 2 Fn}\label{fig:unnamed-chunk-4}
\end{figure}

\begin{figure}

{\centering \includegraphics[width=0.5\linewidth,height=0.2\textheight]{Assignment3/graphs/Newton CES σ=5 _n_4} \includegraphics[width=0.5\linewidth,height=0.2\textheight]{Assignment3/graphs/Newton CES σ=5 _n_6} \includegraphics[width=0.5\linewidth,height=0.2\textheight]{Assignment3/graphs/Newton CES σ=5 _n_11} \includegraphics[width=0.5\linewidth,height=0.2\textheight]{Assignment3/graphs/Newton CES σ=5 _n_21} 

}

\caption{Interpolation Newton CES sigma 5 Fn}\label{fig:unnamed-chunk-5}
\end{figure}

\begin{figure}

{\centering \includegraphics[width=0.5\linewidth,height=0.2\textheight]{Assignment3/graphs/Newton CES σ=10 _n_4} \includegraphics[width=0.5\linewidth,height=0.2\textheight]{Assignment3/graphs/Newton CES σ=10 _n_6} \includegraphics[width=0.5\linewidth,height=0.2\textheight]{Assignment3/graphs/Newton CES σ=10 _n_11} \includegraphics[width=0.5\linewidth,height=0.2\textheight]{Assignment3/graphs/Newton CES σ=10 _n_21} 

}

\caption{Interpolation Newton CES sigma 10 Fn}\label{fig:unnamed-chunk-6}
\end{figure}

\hypertarget{cubic-spline-natural}{%
\subsubsection{Cubic Spline: Natural}\label{cubic-spline-natural}}

\begin{figure}

{\centering \includegraphics[width=0.5\linewidth,height=0.2\textheight]{Assignment3/graphs/CSN_Log _n_4} \includegraphics[width=0.5\linewidth,height=0.2\textheight]{Assignment3/graphs/CSN_Log _n_6} \includegraphics[width=0.5\linewidth,height=0.2\textheight]{Assignment3/graphs/CSN_Log _n_11} \includegraphics[width=0.5\linewidth,height=0.2\textheight]{Assignment3/graphs/CSN_Log _n_21} 

}

\caption{Interpolation Natural Cubic Spline Log Fn}\label{fig:unnamed-chunk-7}
\end{figure}

\begin{figure}

{\centering \includegraphics[width=0.5\linewidth,height=0.2\textheight]{Assignment3/graphs/CSN_Square _n_4} \includegraphics[width=0.5\linewidth,height=0.2\textheight]{Assignment3/graphs/CSN_Square _n_6} \includegraphics[width=0.5\linewidth,height=0.2\textheight]{Assignment3/graphs/CSN_Square _n_11} \includegraphics[width=0.5\linewidth,height=0.2\textheight]{Assignment3/graphs/CSN_Square _n_21} 

}

\caption{Interpolation Natural Cubic Spline Root Square Fn}\label{fig:unnamed-chunk-8}
\end{figure}

\begin{figure}

{\centering \includegraphics[width=0.5\linewidth,height=0.2\textheight]{Assignment3/graphs/CSN_CES σ=2 _n_4} \includegraphics[width=0.5\linewidth,height=0.2\textheight]{Assignment3/graphs/CSN_CES σ=2 _n_6} \includegraphics[width=0.5\linewidth,height=0.2\textheight]{Assignment3/graphs/CSN_CES σ=2 _n_11} \includegraphics[width=0.5\linewidth,height=0.2\textheight]{Assignment3/graphs/CSN_CES σ=2 _n_21} 

}

\caption{Interpolation Natural Cubic Spline CES sigma 2 Fn}\label{fig:unnamed-chunk-9}
\end{figure}

\begin{figure}

{\centering \includegraphics[width=0.5\linewidth,height=0.2\textheight]{Assignment3/graphs/CSN_CES σ=5 _n_4} \includegraphics[width=0.5\linewidth,height=0.2\textheight]{Assignment3/graphs/CSN_CES σ=5 _n_6} \includegraphics[width=0.5\linewidth,height=0.2\textheight]{Assignment3/graphs/CSN_CES σ=5 _n_11} \includegraphics[width=0.5\linewidth,height=0.2\textheight]{Assignment3/graphs/CSN_CES σ=5 _n_21} 

}

\caption{Interpolation Natural Cubic Spline CES sigma 5 Fn}\label{fig:unnamed-chunk-10}
\end{figure}

\begin{figure}

{\centering \includegraphics[width=0.5\linewidth,height=0.2\textheight]{Assignment3/graphs/CSN_CES σ=10 _n_4} \includegraphics[width=0.5\linewidth,height=0.2\textheight]{Assignment3/graphs/CSN_CES σ=10 _n_6} \includegraphics[width=0.5\linewidth,height=0.2\textheight]{Assignment3/graphs/CSN_CES σ=10 _n_11} \includegraphics[width=0.5\linewidth,height=0.2\textheight]{Assignment3/graphs/CSN_CES σ=10 _n_21} 

}

\caption{Interpolation Natural Cubic Spline CES sigma 10 Fn}\label{fig:unnamed-chunk-11}
\end{figure}

\end{document}
