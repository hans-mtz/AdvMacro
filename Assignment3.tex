\PassOptionsToPackage{unicode=true}{hyperref} % options for packages loaded elsewhere
\PassOptionsToPackage{hyphens}{url}
%
\documentclass[]{article}
\usepackage{lmodern}
\usepackage{amssymb,amsmath}
\usepackage{ifxetex,ifluatex}
\usepackage{fixltx2e} % provides \textsubscript
\ifnum 0\ifxetex 1\fi\ifluatex 1\fi=0 % if pdftex
  \usepackage[T1]{fontenc}
  \usepackage[utf8]{inputenc}
  \usepackage{textcomp} % provides euro and other symbols
\else % if luatex or xelatex
  \usepackage{unicode-math}
  \defaultfontfeatures{Ligatures=TeX,Scale=MatchLowercase}
\fi
% use upquote if available, for straight quotes in verbatim environments
\IfFileExists{upquote.sty}{\usepackage{upquote}}{}
% use microtype if available
\IfFileExists{microtype.sty}{%
\usepackage[]{microtype}
\UseMicrotypeSet[protrusion]{basicmath} % disable protrusion for tt fonts
}{}
\IfFileExists{parskip.sty}{%
\usepackage{parskip}
}{% else
\setlength{\parindent}{0pt}
\setlength{\parskip}{6pt plus 2pt minus 1pt}
}
\usepackage{hyperref}
\hypersetup{
            pdftitle={Advanced Macro},
            pdfauthor={Hans Martinez},
            pdfborder={0 0 0},
            breaklinks=true}
\urlstyle{same}  % don't use monospace font for urls
\usepackage[margin=1in]{geometry}
\usepackage{color}
\usepackage{fancyvrb}
\newcommand{\VerbBar}{|}
\newcommand{\VERB}{\Verb[commandchars=\\\{\}]}
\DefineVerbatimEnvironment{Highlighting}{Verbatim}{commandchars=\\\{\}}
% Add ',fontsize=\small' for more characters per line
\usepackage{framed}
\definecolor{shadecolor}{RGB}{248,248,248}
\newenvironment{Shaded}{\begin{snugshade}}{\end{snugshade}}
\newcommand{\AlertTok}[1]{\textcolor[rgb]{0.94,0.16,0.16}{#1}}
\newcommand{\AnnotationTok}[1]{\textcolor[rgb]{0.56,0.35,0.01}{\textbf{\textit{#1}}}}
\newcommand{\AttributeTok}[1]{\textcolor[rgb]{0.77,0.63,0.00}{#1}}
\newcommand{\BaseNTok}[1]{\textcolor[rgb]{0.00,0.00,0.81}{#1}}
\newcommand{\BuiltInTok}[1]{#1}
\newcommand{\CharTok}[1]{\textcolor[rgb]{0.31,0.60,0.02}{#1}}
\newcommand{\CommentTok}[1]{\textcolor[rgb]{0.56,0.35,0.01}{\textit{#1}}}
\newcommand{\CommentVarTok}[1]{\textcolor[rgb]{0.56,0.35,0.01}{\textbf{\textit{#1}}}}
\newcommand{\ConstantTok}[1]{\textcolor[rgb]{0.00,0.00,0.00}{#1}}
\newcommand{\ControlFlowTok}[1]{\textcolor[rgb]{0.13,0.29,0.53}{\textbf{#1}}}
\newcommand{\DataTypeTok}[1]{\textcolor[rgb]{0.13,0.29,0.53}{#1}}
\newcommand{\DecValTok}[1]{\textcolor[rgb]{0.00,0.00,0.81}{#1}}
\newcommand{\DocumentationTok}[1]{\textcolor[rgb]{0.56,0.35,0.01}{\textbf{\textit{#1}}}}
\newcommand{\ErrorTok}[1]{\textcolor[rgb]{0.64,0.00,0.00}{\textbf{#1}}}
\newcommand{\ExtensionTok}[1]{#1}
\newcommand{\FloatTok}[1]{\textcolor[rgb]{0.00,0.00,0.81}{#1}}
\newcommand{\FunctionTok}[1]{\textcolor[rgb]{0.00,0.00,0.00}{#1}}
\newcommand{\ImportTok}[1]{#1}
\newcommand{\InformationTok}[1]{\textcolor[rgb]{0.56,0.35,0.01}{\textbf{\textit{#1}}}}
\newcommand{\KeywordTok}[1]{\textcolor[rgb]{0.13,0.29,0.53}{\textbf{#1}}}
\newcommand{\NormalTok}[1]{#1}
\newcommand{\OperatorTok}[1]{\textcolor[rgb]{0.81,0.36,0.00}{\textbf{#1}}}
\newcommand{\OtherTok}[1]{\textcolor[rgb]{0.56,0.35,0.01}{#1}}
\newcommand{\PreprocessorTok}[1]{\textcolor[rgb]{0.56,0.35,0.01}{\textit{#1}}}
\newcommand{\RegionMarkerTok}[1]{#1}
\newcommand{\SpecialCharTok}[1]{\textcolor[rgb]{0.00,0.00,0.00}{#1}}
\newcommand{\SpecialStringTok}[1]{\textcolor[rgb]{0.31,0.60,0.02}{#1}}
\newcommand{\StringTok}[1]{\textcolor[rgb]{0.31,0.60,0.02}{#1}}
\newcommand{\VariableTok}[1]{\textcolor[rgb]{0.00,0.00,0.00}{#1}}
\newcommand{\VerbatimStringTok}[1]{\textcolor[rgb]{0.31,0.60,0.02}{#1}}
\newcommand{\WarningTok}[1]{\textcolor[rgb]{0.56,0.35,0.01}{\textbf{\textit{#1}}}}
\usepackage{longtable,booktabs}
% Fix footnotes in tables (requires footnote package)
\IfFileExists{footnote.sty}{\usepackage{footnote}\makesavenoteenv{longtable}}{}
\usepackage{graphicx,grffile}
\makeatletter
\def\maxwidth{\ifdim\Gin@nat@width>\linewidth\linewidth\else\Gin@nat@width\fi}
\def\maxheight{\ifdim\Gin@nat@height>\textheight\textheight\else\Gin@nat@height\fi}
\makeatother
% Scale images if necessary, so that they will not overflow the page
% margins by default, and it is still possible to overwrite the defaults
% using explicit options in \includegraphics[width, height, ...]{}
\setkeys{Gin}{width=\maxwidth,height=\maxheight,keepaspectratio}
\setlength{\emergencystretch}{3em}  % prevent overfull lines
\providecommand{\tightlist}{%
  \setlength{\itemsep}{0pt}\setlength{\parskip}{0pt}}
\setcounter{secnumdepth}{0}
% Redefines (sub)paragraphs to behave more like sections
\ifx\paragraph\undefined\else
\let\oldparagraph\paragraph
\renewcommand{\paragraph}[1]{\oldparagraph{#1}\mbox{}}
\fi
\ifx\subparagraph\undefined\else
\let\oldsubparagraph\subparagraph
\renewcommand{\subparagraph}[1]{\oldsubparagraph{#1}\mbox{}}
\fi

% set default figure placement to htbp
\makeatletter
\def\fps@figure{htbp}
\makeatother

\usepackage{etoolbox}
\makeatletter
\providecommand{\subtitle}[1]{% add subtitle to \maketitle
  \apptocmd{\@title}{\par {\large #1 \par}}{}{}
}
\makeatother

\title{Advanced Macro}
\providecommand{\subtitle}[1]{}
\subtitle{Assignment 3}
\author{Hans Martinez}
\date{Oct 12, 2020}

\begin{document}
\maketitle

\hypertarget{interpolation}{%
\section{Interpolation}\label{interpolation}}

\href{https://github.com/hans-mtz/AdvMacro/blob/master/A3.jl}{Julia
code: click here.}

\begin{Shaded}
\begin{Highlighting}[]
\CommentTok{# Julia code}
\CommentTok{# See A3.jl}
\end{Highlighting}
\end{Shaded}

\hypertarget{log-utility}{%
\subsection{Log utility}\label{log-utility}}

\begin{figure}

{\centering \includegraphics[width=0.5\linewidth,height=0.2\textheight]{Assignment3/graphs/Log 4} \includegraphics[width=0.5\linewidth,height=0.2\textheight]{Assignment3/graphs/Log 6} \includegraphics[width=0.5\linewidth,height=0.2\textheight]{Assignment3/graphs/Log 11} \includegraphics[width=0.5\linewidth,height=0.2\textheight]{Assignment3/graphs/Log 21} 

}

\caption{Interpolation Log Fn}\label{fig:unnamed-chunk-2}
\end{figure}

\hypertarget{square-root-utility}{%
\subsection{Square root utility}\label{square-root-utility}}

\begin{figure}

{\centering \includegraphics[width=0.5\linewidth,height=0.2\textheight]{Assignment3/graphs/SQR 4} \includegraphics[width=0.5\linewidth,height=0.2\textheight]{Assignment3/graphs/SQR 6} \includegraphics[width=0.5\linewidth,height=0.2\textheight]{Assignment3/graphs/SQR 11} \includegraphics[width=0.5\linewidth,height=0.2\textheight]{Assignment3/graphs/SQR 21} 

}

\caption{Interpolation Square Root Fn}\label{fig:unnamed-chunk-3}
\end{figure}

\hypertarget{crra-utility}{%
\subsection{CRRA utility}\label{crra-utility}}

\(\sigma = 2\)

\begin{figure}

{\centering \includegraphics[width=0.5\linewidth,height=0.2\textheight]{Assignment3/graphs/CES_2 4} \includegraphics[width=0.5\linewidth,height=0.2\textheight]{Assignment3/graphs/CES_2 6} \includegraphics[width=0.5\linewidth,height=0.2\textheight]{Assignment3/graphs/CES_2 11} \includegraphics[width=0.5\linewidth,height=0.2\textheight]{Assignment3/graphs/CES_2 21} 

}

\caption{Interpolation CRRA sigma=2 Fn}\label{fig:unnamed-chunk-4}
\end{figure}

\(\sigma = 5\)

\begin{figure}

{\centering \includegraphics[width=0.5\linewidth,height=0.2\textheight]{Assignment3/graphs/CES_5 4} \includegraphics[width=0.5\linewidth,height=0.2\textheight]{Assignment3/graphs/CES_5 6} \includegraphics[width=0.5\linewidth,height=0.2\textheight]{Assignment3/graphs/CES_5 11} \includegraphics[width=0.5\linewidth,height=0.2\textheight]{Assignment3/graphs/CES_5 21} 

}

\caption{Interpolation CRRA sigma=5 Fn}\label{fig:unnamed-chunk-5}
\end{figure}

\(\sigma = 10\)

\begin{figure}

{\centering \includegraphics[width=0.5\linewidth,height=0.2\textheight]{Assignment3/graphs/CES_10 4} \includegraphics[width=0.5\linewidth,height=0.2\textheight]{Assignment3/graphs/CES_10 6} \includegraphics[width=0.5\linewidth,height=0.2\textheight]{Assignment3/graphs/CES_10 11} \includegraphics[width=0.5\linewidth,height=0.2\textheight]{Assignment3/graphs/CES_10 21} 

}

\caption{Interpolation CRRA sigma=10 Fn}\label{fig:unnamed-chunk-6}
\end{figure}

\hypertarget{summary-of-interpolation-errors}{%
\subsection{Summary of interpolation
errors}\label{summary-of-interpolation-errors}}

I'm using the Eucledian Norm of the differences between the original
function and the interpolation method. Specifically
\[ || f(x)-q(x) ||^2 = \left[ \sum_i \left( f(x_i)-q(x_i) \right)^2 \right]^{\frac{1}{2}} \]

\begin{longtable}[]{@{}rrrrr@{}}
\toprule
Log & n=4 & n=6 & n=11 & n=21\tabularnewline
\midrule
\endhead
Newton Basis Polynomials & \(6.286\) & \(2.28\) & \(0.4082\) &
\(0.03642\)\tabularnewline
Natural Cubic Splines & \(8.375\) & \(4.379\) & \(1.606\) &
\(0.4928\)\tabularnewline
Shape-preserving Schumaker Splines & \(1.176\) & \(0.533\) & \(0.1543\)
& \(0.03563\)\tabularnewline
\bottomrule
\end{longtable}

\begin{longtable}[]{@{}rrrrr@{}}
\toprule
Square Root & n=4 & n=6 & n=11 & n=21\tabularnewline
\midrule
\endhead
Newton Basis Polynomials & \(0.579\) & \(0.166\) & \(0.02202\) &
\(0.001462\)\tabularnewline
Natural Cubic Splines & \(0.8726\) & \(0.3989\) & \(0.1245\) &
\(0.03344\)\tabularnewline
Shape-preserving Schumaker Splines & \(0.098\) & \(0.03925\) &
\(0.00975\) & \(0.001983\)\tabularnewline
\bottomrule
\end{longtable}

\begin{longtable}[]{@{}rrrrr@{}}
\toprule
CES \(\sigma = 2\) & n=4 & n=6 & n=11 & n=21\tabularnewline
\midrule
\endhead
Newton Basis Polynomials & \(98.2\) & \(48.55\) & \(13.68\) &
\(2.012\)\tabularnewline
Natural Cubic Splines & \(117.2\) & \(73.55\) & \(34.44\) &
\(13.26\)\tabularnewline
Shape-preserving Schumaker Splines & \(30.79\) & \(16.31\) & \(5.628\) &
\(1.481\)\tabularnewline
\bottomrule
\end{longtable}

\begin{longtable}[]{@{}rrrrr@{}}
\toprule
\begin{minipage}[b]{0.25\columnwidth}\raggedleft
CES \(\sigma = 5\)\strut
\end{minipage} & \begin{minipage}[b]{0.15\columnwidth}\raggedleft
n=4\strut
\end{minipage} & \begin{minipage}[b]{0.15\columnwidth}\raggedleft
n=6\strut
\end{minipage} & \begin{minipage}[b]{0.15\columnwidth}\raggedleft
n=11\strut
\end{minipage} & \begin{minipage}[b]{0.15\columnwidth}\raggedleft
n=21\strut
\end{minipage}\tabularnewline
\midrule
\endhead
\begin{minipage}[t]{0.25\columnwidth}\raggedleft
Newton Basis Polynomials\strut
\end{minipage} & \begin{minipage}[t]{0.15\columnwidth}\raggedleft
\(3.195 \cdot 10^{5}\)\strut
\end{minipage} & \begin{minipage}[t]{0.15\columnwidth}\raggedleft
\(2.074 \cdot 10^{5}\)\strut
\end{minipage} & \begin{minipage}[t]{0.15\columnwidth}\raggedleft
\(1.024 \cdot 10^{5}\)\strut
\end{minipage} & \begin{minipage}[t]{0.15\columnwidth}\raggedleft
\(3.559 \cdot 10^{4}\)\strut
\end{minipage}\tabularnewline
\begin{minipage}[t]{0.25\columnwidth}\raggedleft
Natural Cubic Splines\strut
\end{minipage} & \begin{minipage}[t]{0.15\columnwidth}\raggedleft
\(3.592 \cdot 10^{5}\)\strut
\end{minipage} & \begin{minipage}[t]{0.15\columnwidth}\raggedleft
\(2.647 \cdot 10^{5}\)\strut
\end{minipage} & \begin{minipage}[t]{0.15\columnwidth}\raggedleft
\(1.655 \cdot 10^{5}\)\strut
\end{minipage} & \begin{minipage}[t]{0.15\columnwidth}\raggedleft
\(9.166 \cdot 10^{4}\)\strut
\end{minipage}\tabularnewline
\begin{minipage}[t]{0.25\columnwidth}\raggedleft
Shape-preserving Schumaker Splines\strut
\end{minipage} & \begin{minipage}[t]{0.15\columnwidth}\raggedleft
\(1.419 \cdot 10^{5}\)\strut
\end{minipage} & \begin{minipage}[t]{0.15\columnwidth}\raggedleft
\(9.976 \cdot 10^{4}\)\strut
\end{minipage} & \begin{minipage}[t]{0.15\columnwidth}\raggedleft
\(5.497 \cdot 10^{4}\)\strut
\end{minipage} & \begin{minipage}[t]{0.15\columnwidth}\raggedleft
\(2.357 \cdot 10^{4}\)\strut
\end{minipage}\tabularnewline
\bottomrule
\end{longtable}

\begin{longtable}[]{@{}rrrrr@{}}
\toprule
\begin{minipage}[b]{0.24\columnwidth}\raggedleft
CES \(\sigma = 10\)\strut
\end{minipage} & \begin{minipage}[b]{0.15\columnwidth}\raggedleft
n=4\strut
\end{minipage} & \begin{minipage}[b]{0.15\columnwidth}\raggedleft
n=6\strut
\end{minipage} & \begin{minipage}[b]{0.15\columnwidth}\raggedleft
n=11\strut
\end{minipage} & \begin{minipage}[b]{0.15\columnwidth}\raggedleft
n=21\strut
\end{minipage}\tabularnewline
\midrule
\endhead
\begin{minipage}[t]{0.24\columnwidth}\raggedleft
Newton Basis Polynomials\strut
\end{minipage} & \begin{minipage}[t]{0.15\columnwidth}\raggedleft
\(4.805 \cdot 10^{11}\)\strut
\end{minipage} & \begin{minipage}[t]{0.15\columnwidth}\raggedleft
\(3.295 \cdot 10^{11}\)\strut
\end{minipage} & \begin{minipage}[t]{0.15\columnwidth}\raggedleft
\(1.908 \cdot 10^{11}\)\strut
\end{minipage} & \begin{minipage}[t]{0.15\columnwidth}\raggedleft
\(9.752 \cdot 10^{10}\)\strut
\end{minipage}\tabularnewline
\begin{minipage}[t]{0.24\columnwidth}\raggedleft
Natural Cubic Splines\strut
\end{minipage} & \begin{minipage}[t]{0.15\columnwidth}\raggedleft
\(5.35 \cdot 10^{11}\)\strut
\end{minipage} & \begin{minipage}[t]{0.15\columnwidth}\raggedleft
\(4.067 \cdot 10^{11}\)\strut
\end{minipage} & \begin{minipage}[t]{0.15\columnwidth}\raggedleft
\(2.748 \cdot 10^{11}\)\strut
\end{minipage} & \begin{minipage}[t]{0.15\columnwidth}\raggedleft
\(1.768 \cdot 10^{11}\)\strut
\end{minipage}\tabularnewline
\begin{minipage}[t]{0.24\columnwidth}\raggedleft
Shape-preserving Schumaker Splines\strut
\end{minipage} & \begin{minipage}[t]{0.15\columnwidth}\raggedleft
\(2.214 \cdot 10^{11}\)\strut
\end{minipage} & \begin{minipage}[t]{0.15\columnwidth}\raggedleft
\(1.661 \cdot 10^{11}\)\strut
\end{minipage} & \begin{minipage}[t]{0.15\columnwidth}\raggedleft
\(1.08 \cdot 10^{11}\)\strut
\end{minipage} & \begin{minipage}[t]{0.15\columnwidth}\raggedleft
\(6.367 \cdot 10^{10}\)\strut
\end{minipage}\tabularnewline
\bottomrule
\end{longtable}

\begin{figure}

{\centering \includegraphics[width=0.5\linewidth,height=0.2\textheight]{Assignment3/graphs/IntError_Log} \includegraphics[width=0.5\linewidth,height=0.2\textheight]{Assignment3/graphs/IntError_SQR} \includegraphics[width=0.5\linewidth,height=0.2\textheight]{Assignment3/graphs/IntError_CES_2} \includegraphics[width=0.5\linewidth,height=0.2\textheight]{Assignment3/graphs/IntError_CES_5} \includegraphics[width=0.5\linewidth,height=0.2\textheight]{Assignment3/graphs/IntError_CES_10} 

}

\caption{Interpolation error of differet interpolation methods}\label{fig:unnamed-chunk-7}
\end{figure}

\hypertarget{curvature}{%
\section{Curvature}\label{curvature}}

\begin{figure}

{\centering \includegraphics[width=0.5\linewidth,height=0.2\textheight]{Assignment3/graphs/Log 1.0} \includegraphics[width=0.5\linewidth,height=0.2\textheight]{Assignment3/graphs/Log 1.5} \includegraphics[width=0.5\linewidth,height=0.2\textheight]{Assignment3/graphs/Log 2.0} \includegraphics[width=0.5\linewidth,height=0.2\textheight]{Assignment3/graphs/Log 3.0} 

}

\caption{Interpolation, differet methods, n=6, log function.}\label{fig:unnamed-chunk-8}
\end{figure}

\begin{figure}

{\centering \includegraphics[width=0.5\linewidth,height=0.2\textheight]{Assignment3/graphs/SQR 1.0} \includegraphics[width=0.5\linewidth,height=0.2\textheight]{Assignment3/graphs/SQR 1.5} \includegraphics[width=0.5\linewidth,height=0.2\textheight]{Assignment3/graphs/SQR 2.0} \includegraphics[width=0.5\linewidth,height=0.2\textheight]{Assignment3/graphs/SQR 3.0} 

}

\caption{Interpolation, differet methods, n=6, square root function.}\label{fig:unnamed-chunk-9}
\end{figure}

\begin{figure}

{\centering \includegraphics[width=0.5\linewidth,height=0.2\textheight]{Assignment3/graphs/CES_2 1.0} \includegraphics[width=0.5\linewidth,height=0.2\textheight]{Assignment3/graphs/CES_2 1.5} \includegraphics[width=0.5\linewidth,height=0.2\textheight]{Assignment3/graphs/CES_2 2.0} \includegraphics[width=0.5\linewidth,height=0.2\textheight]{Assignment3/graphs/CES_2 3.0} 

}

\caption{Interpolation, differet methods, n=6, CES sigma=2 function.}\label{fig:unnamed-chunk-10}
\end{figure}

\begin{figure}

{\centering \includegraphics[width=0.5\linewidth,height=0.2\textheight]{Assignment3/graphs/CES_5 1.0} \includegraphics[width=0.5\linewidth,height=0.2\textheight]{Assignment3/graphs/CES_5 1.5} \includegraphics[width=0.5\linewidth,height=0.2\textheight]{Assignment3/graphs/CES_5 2.0} \includegraphics[width=0.5\linewidth,height=0.2\textheight]{Assignment3/graphs/CES_5 3.0} 

}

\caption{Interpolation, differet methods, n=6, CES sigma=5 function.}\label{fig:unnamed-chunk-11}
\end{figure}

\begin{figure}

{\centering \includegraphics[width=0.5\linewidth,height=0.2\textheight]{Assignment3/graphs/CES_10 1.0} \includegraphics[width=0.5\linewidth,height=0.2\textheight]{Assignment3/graphs/CES_10 1.5} \includegraphics[width=0.5\linewidth,height=0.2\textheight]{Assignment3/graphs/CES_10 2.0} \includegraphics[width=0.5\linewidth,height=0.2\textheight]{Assignment3/graphs/CES_10 3.0} 

}

\caption{Interpolation, differet methods, n=6, CES sigma=10 function.}\label{fig:unnamed-chunk-12}
\end{figure}

\hypertarget{interpolation-error}{%
\subsection{Interpolation error}\label{interpolation-error}}

\begin{figure}

{\centering \includegraphics[width=0.5\linewidth,height=0.2\textheight]{Assignment3/graphs/IntError_Curv_Log} \includegraphics[width=0.5\linewidth,height=0.2\textheight]{Assignment3/graphs/IntError_Curv_SQR} \includegraphics[width=0.5\linewidth,height=0.2\textheight]{Assignment3/graphs/IntError_Curv_CES_2} \includegraphics[width=0.5\linewidth,height=0.2\textheight]{Assignment3/graphs/IntError_Curv_CES_5} \includegraphics[width=0.5\linewidth,height=0.2\textheight]{Assignment3/graphs/IntError_Curv_CES_10} 

}

\caption{Interpolation error of differet methods, varying the curvature of the grid.}\label{fig:unnamed-chunk-13}
\end{figure}

\end{document}
